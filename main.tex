\documentclass[letterpaper,10pt]{article}
\usepackage{fancyhdr}
\usepackage{graphicx}
\usepackage{ragged2e}
\usepackage{blindtext}
\usepackage{biblatex}
\usepackage[german]{babel}
\usepackage[headsep=2cm,headheight=1cm]{geometry}

\graphicspath{ {./images/} }
\addbibresource{literature.bib}




\begin{document}

\newgeometry{ bottom=10mm, left=5mm, right=5mm, showframe}
\setlength{\headheight}{14pt}
\setlength{\hsize}{0.9\hsize}% emphasize effects

\pagestyle{fancy}
%\fancyhead[C]{
%    \begin{tabular}{c c c}
%        \includegraphics[width=2cm]{2022-05-18-08-39-55} & \huge{HTBLVA, Wien V, Spengergasse}  & \includegraphics[width=2cm]{2022-05-18-08-40-12} \\
%        & \textbf{Höhere Lehranstalt für xxxx} & \\
%        & Ausbildungsschwerpunkt xxxx &
%        
%    \end{tabular}
%}



\begin{center}
    \Huge{Diplomarbeit} \\
    
\end{center}
\begin{center}   
Gesamtprojekt

\end{center}

\begin{center}
    \huge{Die Lösung eines Problems}
\end{center}



\raggedright{
    \begin{tabular}{l l l l} 
        \textbf{Individualles Thema lt. Einreichung hier einsetzen.} & & & \\
        Max Mustermann & 5xHIF & Betreuer: & Alois B. Treuer \\ 
        \\
        \textbf{Individualles Thema lt. Einreichung hier einsetzen.} & & & \\
        Maria Musterfrau & 5xHIF & Betreuer: & Alois B. Treuer \\ 
        \\
        \textbf{Individualles Thema lt. Einreichung hier einsetzen.} & & & \\
        David Muster & 5xHIF & Betreuer: & Alois B. Treuer \\ 

        \\
        \textbf{Individualles Thema lt. Einreichung hier einsetzen.} & & & \\
        Julia Muster & 5xHIF & Betreuer: & Alois B. Treuer \\ 
        &&& Johanna Lehrer
    \end{tabular}

}

Schuljahr 2021/22 \\

Abgabevermerk: \\

Datum: TT.MM.JJJJ

\raggedleft{übernommen von:}

\raggedright

% todo move to include
\pagebreak

Beim zweiseitigem Druck bleibt diese Seite leer. Ansonsten sollte sie gelöscht werden.

% todo move to include
\pagebreak


\begin{center}
    \Huge{EIDESSTATTLICHE ERKLÄRUNG}

\end{center}

Ich erkläre an Eides statt, dass ich die vorliegende Diplomarbeit selbständig und ohne fremde Hilfe verfasst, andere als die angegebenen Quellen und Hilfsmittel nicht benutzt und die den benutzten Quellen wörtlich und inhaltlich entnommenen Stellen als solche erkenntlich gemacht habe.

%% todo move date into variable
\begin{tabular*}{\textwidth}{l l}
    Wien, am TT.MM.JJJJ & Verfasser \& Verfasserinnen: \\
    \\
    \\
    & (Unterschrift Max Mustermann 1) \\
    \\
    & Max Mustermann 1 \\
    \\
    \\
    & (Unterschrift Max Mustermann 2) \\
    \\
    & Max Mustermann 2 \\
    \\
    \\
    & (Unterschrift Max Mustermann 3) \\
    \\
    & Max Mustermann 3 \\
    \\
    \\
    & (Unterschrift Max Mustermann 4) \\
    \\
    & Max Mustermann 4 \\

\end{tabular*}


\pagebreak

% todo move to include

\begin{center}
    \textbf{\Huge{Diplomarbeit}} \\
    \huge{Dokumentation}
\end{center}


\begin{tabular*}{\textwidth}{|l | l|}
    \hline
    Namen der Verfasse/innen: & \\
    \hline
    Jahrgang Schuljahr & \\
    \hline
    Thema der Diplomarbeit & \\
    \hline
    Kooperationspartner \\
    \hline
\end{tabular*}


\begin{tabular*}{\textwidth}{|l | l|}
    \hline
    Aufgabenstellung & \\

    \hline
\end{tabular*}

\begin{tabular*}{\textwidth}{|l | l|}
    \hline
    Realisierung & \\

    \hline
\end{tabular*}


\begin{tabular*}{\textwidth}{|l | l|}
    \hline
    Ergebnisse & \\

    \hline
\end{tabular*}


\begin{tabular*}{\textwidth}{|l | l|}
    \hline
    Typische Grafik, Foto etc. (mit Erkläuterung) & \\

    \hline
\end{tabular*}


\begin{tabular*}{\textwidth}{|l | l|}
    \hline
    Teilnahme an Wettbewerben, Auszeichnungen & \\

    \hline
\end{tabular*}


\begin{tabular*}{\textwidth}{|l | l|}
    \hline
    Möglichkeiten der Einsichtnahme in die Arbeit & \\

    \hline
\end{tabular*}


\begin{tabular*}{\textwidth}{|l | l| l |}
    \hline
    Approbation(Datum / Unterschrift) & & \\
    \hline
     & & \\
    & Prüfer/Prüferin & Direktor \\
    \hline
\end{tabular*}

% das selbe nochmal auf englisch
\pagebreak

\begin{center}
    \textbf{\Huge{Diploma Thesis}} \\
    \huge{Documentation}
\end{center}

\begin{tabular*}{\textwidth}{|l | l|}
    \hline
    Author(s) & \\
    \hline
    Form Academic Year & \\
    \hline
    Topic & \\
    \hline
    cooperation partner \\
    \hline
\end{tabular*}


\begin{tabular*}{\textwidth}{|l | l|}
    \hline
    Assignment of tasks & \\

    \hline
\end{tabular*}

\begin{tabular*}{\textwidth}{|l | l|}
    \hline
    Realisation & \\

    \hline
\end{tabular*}


\begin{tabular*}{\textwidth}{|l | l|}
    \hline
    Results & \\

    \hline
\end{tabular*}


\begin{tabular*}{\textwidth}{|l | l|}
    \hline
    Illustrative graph, photo
(incl. explanation) & \\

    \hline
\end{tabular*}


\begin{tabular*}{\textwidth}{|l | l|}
    \hline
    Participation in competitions,awards & \\

    \hline
\end{tabular*}


\begin{tabular*}{\textwidth}{|l | l|}
    \hline
    Accessibility of diploma thesis & \\

    \hline
\end{tabular*}


\begin{tabular*}{\textwidth}{|l | l| l |}
    \hline
    Approval (date / signature) & & \\
    \hline
     & & \\
    & Examiner & Head of College/Department \\
    \hline
\end{tabular*}

% todo move each chapter to seperate tex file
\pagebreak

\tableofcontents

\section{Einleitung}

Zielsetzung und Aufgabenstellung des Gesamtprojekts,
fachliches und wirtschaftliches Umfeld

Testzitat
 $E = mc^2$ \cite{einstein}

\section{Aufgabenstellung xxxx von Max Mustermann}
\subsection{Grundlagen}
Begriffserklärungen, allgemeine theoretische Grundlagen, Analyse der Ist-Situation
\subsection{Lösungsansätze}
mögliche Lösungsansätze, Begründung des gewählten Lösungsansatzes
\subsection{Ergebnisse}
Konkrete Umsetzung der Lösung im Projekt, Empirische Befunde aus dem Projekt, Lessons Learned
\subsection{Quellen- und Literaturverzeichnis}
Gemäß Zitierregeln, Kann auch für alle Aufgabenstellungen gemeinsam geführt werden
\subsection{Verzeichnis der Abbildungen, Tabellen und Abkürzungen}
Kann auch für alle Aufgabenstellungen gemeinsam geführt werden
\subsection{Begleitprotokoll gemäß § 9 Abs. 2 PrO}
Begleitprotokoll von Max Mustermann mit Datum, Art der Tätigkeit (Bei Besprechung ggf. Verweis auf Besprechungsprotokoll), Aufwand in Stunden (auf 0,25 Std. =15 Minuten genau), verwendete Hilfsmittel

\pagebreak

\section{Aufgabenstellung xxxx von Max Mustermann}
\subsection{Grundlagen}
Begriffserklärungen, allgemeine theoretische Grundlagen, Analyse der Ist-Situation
\subsection{Lösungsansätze}
mögliche Lösungsansätze, Begründung des gewählten Lösungsansatzes
\subsection{Ergebnisse}
Konkrete Umsetzung der Lösung im Projekt, Empirische Befunde aus dem Projekt, Lessons Learned
\subsection{Quellen- und Literaturverzeichnis}
Gemäß Zitierregeln, Kann auch für alle Aufgabenstellungen gemeinsam geführt werden
\subsection{Verzeichnis der Abbildungen, Tabellen und Abkürzungen}
Kann auch für alle Aufgabenstellungen gemeinsam geführt werden
\subsection{Begleitprotokoll gemäß § 9 Abs. 2 PrO}
Begleitprotokoll von Max Mustermann mit Datum, Art der Tätigkeit (Bei Besprechung ggf. Verweis auf Besprechungsprotokoll), Aufwand in Stunden (auf 0,25 Std. =15 Minuten genau), verwendete Hilfsmittel

\pagebreak

\pagebreak

\section{Aufgabenstellung xxxx von Max Mustermann}
\subsection{Grundlagen}
Begriffserklärungen, allgemeine theoretische Grundlagen, Analyse der Ist-Situation
\subsection{Lösungsansätze}
mögliche Lösungsansätze, Begründung des gewählten Lösungsansatzes
\subsection{Ergebnisse}
Konkrete Umsetzung der Lösung im Projekt, Empirische Befunde aus dem Projekt, Lessons Learned
\subsection{Quellen- und Literaturverzeichnis}
Gemäß Zitierregeln, Kann auch für alle Aufgabenstellungen gemeinsam geführt werden
\subsection{Verzeichnis der Abbildungen, Tabellen und Abkürzungen}
Kann auch für alle Aufgabenstellungen gemeinsam geführt werden
\subsection{Begleitprotokoll gemäß § 9 Abs. 2 PrO}
Begleitprotokoll von Max Mustermann mit Datum, Art der Tätigkeit (Bei Besprechung ggf. Verweis auf Besprechungsprotokoll), Aufwand in Stunden (auf 0,25 Std. =15 Minuten genau), verwendete Hilfsmittel

\pagebreak

\section{Quellen- und Literaturverzeichnis}
\printbibliography


\pagebreak

\section{Verzeichnis der Abbildungen, Tabellen und Abkürzungen}

\listoffigures

\listoftables

\pagebreak

\section{Anhang}
Projektdokumentation (PHB gemäß IPMA-Vorlage mit Kostendarstellung, Projektplanung, Besprechungsprotokolle, Abnahmeprotokoll, etc.)
Anforderungsdokumentation \&  Technische Dokumentation (Lastenheft, Product Backlog, Pflichtenheft, Installationsanleitungen, Testprotokolle, …)


\end{document}